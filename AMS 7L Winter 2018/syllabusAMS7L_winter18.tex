\documentclass[11pt]{article}
\usepackage{fullpage,amssymb}
\pagestyle{empty}
\usepackage{color}
\usepackage[colorlinks=true,urlcolor=blue]{hyperref}


\begin{document}
\begin{center}
{\Large {\bf AMS 7L \hfill Computer Lab for Statistics (AMS 7) \hfill Winter 2018}
}

\vspace{.3in}
{\Large {\bf Course Policies and Syllabus}}
\vspace{.25in}

\begin{tabular}{rcc} 
{\bf Instructors:}   &  Raquel Barata & Dan Spencer \\
{\bf Email:} & rbarata@ucsc.edu & daspence@ucsc.edu
\end{tabular}
\end{center}

\vspace{.2in}

\noindent {\bf Computing Lab Hours:} \\
\begin{tabular}{l l l l l}
	MW & 9:00 -- 10:45am & Soc Sci 1 135 & Spencer & *** \\
	MW & 11:00am -- 12:45pm & Soc Sci 1 135 & Spencer & ***\\
	MW & 1:00 -- 2:45pm & Soc Sci 1 135 & Spencer & ***\\
	MW & 3:00 -- 4:45pm & Soc Sci 1 135 & Spencer & ***\\
	MW & 5:00 -- 6:45 pm & Soc Sci 1 135 & Barata & ***  \\
	TuTh & 9:00 -- 10:45 am & Soc Sci 1 135 & Barata & ***\\
	TuTh & 11:00 am -- 12:45 pm& Soc Sci 1 135 & Barata & *** \\
	TuTh & 7:00 -- 8:45 pm & Soc Sci 1 135 & Barata & *** \\
\end{tabular}

\vspace{0.15in}

\noindent*** These labs will only be attended by an instructor the first mandatory occurrence on either January 16th or 17th. After that, you will need to attend a drop-in lab for help. If no drop-in lab fits your schedule, contact Raquel or Dan to schedule an appointment.
\vspace{0.3in}

\noindent {\bf Drop-in Lab Times / Office Hours:} \\
\begin{tabular}{l l l l l}
	Wednesday & 9:00 -- 10:45am & Soc Sci 1 135 & Spencer & \\
	Wednesday & 11:00am -- 12:45pm & Soc Sci 1 135 & Spencer &\\
	Wednesday& 1:00 -- 2:45pm & Soc Sci 1 135 & Spencer &\\
	Wednesday & 3:00 -- 4:45pm & Soc Sci 1 135 & Barata &\\
	Wednesday & 5:00 -- 6:45 pm & Soc Sci 1 135 & Barata &   \\
	Monday & 5:00 -- 6:45 pm & Soc Sci 1 135 & Barata &  \\
\end{tabular}
\vspace{0.15in}



\noindent Attendance is NOT required for drop-in lab sessions after the first meeting. 
\vspace{0.15in}

\noindent {\bf Web page:} All announcements and lab assignments are in {\em Canvas}. Login to your {\em Canvas} using your GoldID and password and enter AMS 7L Winter 2017. 
The login page for {\em Canvas} can be accessed using the following URL: 
\begin{center}\url{http://canvas.ucsc.edu}\end{center} 

\vspace{0.3in}

\noindent {\bf Associated Lectures:} \\
Matthew Heiner, MWF 2:40 -- 3:45pm, J Bask Aud 101 \\
Immanuel Williams, TuTh 5:20 -- 6:55 pm, J Bask Aud 101

\vspace{.3in}

%\noindent {\bf Text: }{\em Biostatistics for the Biological
%  and Health Sciences},
%M. M. Triola and M. F. Triola, Pearson (2006), or recommended text for AMS7

\noindent {\bf Text: }{\em Biostatistics for the Biological
	and Health Sciences (2nd Edition)},
M.\ M.\ Triola, M.\ F.\ Triola, J.\ Roy. Pearson (2017)

\vspace{.3in}

\noindent {\bf Course Objectives:} To acquire the technological skills
needed to implement methods learned in AMS 7 using the statistical
software \href{https://www.jmp.com/en_us/home.html}{JMP}, and to reinforce various concepts from AMS 7 through
computer simulation and data analysis.  \vspace{.2in}

\noindent {\bf Lab Assignments:} Lab assignments will be completed,
submitted, and reviewed in {\em Canvas}.  The labs will be posted in the
{\em Quizzes} section. \\
 
A (single) lab assignment will be posted and due once a week. Students are allowed 2 attempts and the highest score will be recorded. Labs will be posted every Monday at 9:00 am and due the following Monday at
9:00 am with the exception of the last lab. 
Lab 10 will be due on the last Friday of the Winter finals week, 3/23/2018 at 9:00 am. \\

Labs are self-paced and do NOT have a time limit; however, ALL labs
MUST be submitted by the posted due date. You do not have to complete
lab assignments in one session. You can save
assignments in {\em Canvas} and return to complete them at a later time.
Most lab assignments will consist of multiple sections, each of which
you will be expected to complete, submit, and review one at a time
BEFORE starting the next section of the lab.  Labs are designed to
take approximately 90 minutes to complete all parts combined, but may
be shorter or longer depending on your familiarity with the
material. You are allowed and encouraged to work on labs alongside
your peers, but every student is expected to do their own calculations
and JMP analysis required by the lab. Submitting work not completed by you is a violation of academic integrity. \\

In the event a student is found in violation of the UCSC Academic Integrity policy, he or she may face both academic sanctions imposed by the instructor of record and disciplinary sanctions imposed either by the provost of his or her college or the Academic Tribunal convened to hear the case. Violations of the Academic Integrity policy can result in dismissal from the university and a permanent notation on a student's transcript.
For the full policy and disciplinary procedures on academic dishonesty, students and instructors should refer to the Academic Integrity page \url{https://www.ue.ucsc.edu/academic_misconduct} at the Division of Undergraduate Education.
\vspace{0.2in}

\noindent {\bf Late Work:} Late submissions will {\bf NOT} be
accepted. The class accommodates missing an entire Lab assignment by
taking the highest 9 out of 10 Labs (see {\bf Course Grade} section
below). {\bf Therefore, instructors will adhere to a strict assignment
submission policy. } Complete the Labs early in the week. Do not wait
until the day the assignments are due!  In cases of extenuating
circumstances, accommodating late work will be left at the discretion of the instructors. In such cases, email both instructors at least 48 hours before the due date of the assignment. Documentation of the extenuating circumstance (for medical emergency, etc) may be required.\\
 
% \vspace{.2in}
 \noindent {\bf Student Support:} Students are encouraged to email
 instructors at any time throughout the course. Note that last minute
 emails may not be answered immediately. Thus, be sure to send your
 inquiries to instructors well before the due date (don't wait until
 the night before to do the lab). In person appointments may be
 scheduled if additional help is needed. \\

\noindent All data files used in labs can be found in {\em Files} in {\em Canvas}.\\ 

\newpage
\noindent {\bf Schedule and Content List:}  \vspace{.05in}  \\ 
\indent \begin{tabular}{| l | l |p{10cm}|} \hline
Lab \# & Due Date & Content \\ \hline
Lab 1 & Jan.\ 22, 9 am &Practice with Data Types, Starting JMP. \\ \hline
Lab 2 & Jan.\ 29, 9 am & Looking at data. Measures of central tendency, Measures of dispersion. \\ \hline

Lab 3 & Feb.\ 5, 9 am &Relative Frequency, Probability (including Bayes Theorem), Binomial and Poisson distribution.\\ \hline

Lab 4 & Feb.\ 12, 9 am & Means of Normals, Central Limit Theorem, Normal Approximation to Binomial\\ \hline

Lab 5 & Feb.\ 19, 9 am & Review lab.\\ \hline

Lab 6 & Feb.\ 26, 9 am &Confidence Intervals for Means, Confidence Intervals for Proportions.\\ \hline

Lab 7 & Mar.\ 5, 9 am &One Sample Hypothesis Tests for Means, Hypothesis Tests for Proportions. Two-sample Tests for Means.\\ \hline

Lab 8&  Mar.\ 12, 9 am &Regression, Residuals and Transformations\\ \hline

Lab 9& Mar.\ 19, 9 am & Multiple Regression, Goodness-of-Fit Tests\\ \hline

Lab 10 & Mar.\ 23, 9 am & Optional lab: Polynomial Regression, Optimization.\\ \hline
\end{tabular}

\vspace{.3in}

\noindent {\bf Course Grade:} Grades will be based on a point
system.  Labs are worth 100 points each.  40 points will be awarded
simply for completing (and submitting) ALL sections of a lab
assignment.  Students may receive 20 completion points if AT LEAST
half the lab sections are completed (eg. 2 out of 3, 2 out of 4, 3 out
of 5, etc.). The remaining 60 points will be awarded for correct
answers.  The first nine labs are required, so the total number of
points for the course is 900. A tenth lab assignment will be available
to replace your lowest Lab score.  The same rubric applies to Lab 10:
40 points for completing ALL elements of the lab and 60 points for correct answers.  The percentage of the total points
earned (out of 900) will determine a student's letter grade: 90\% -
100\% is an A, 80\% - 89\% is a B, 70\% - 79\% is a C, 60\% - 69\% is
a D, and 0 - 59\% is an F.  Note that A+ will not be given for people
who earn more than 900 total points. \\

\end{document}
